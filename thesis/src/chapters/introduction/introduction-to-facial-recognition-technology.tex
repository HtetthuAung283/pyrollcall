\section{Introduction to Facial Recognition Technology}
Facial Recognition is a technology which is capable of identifying or verifying the individuals in a digital image or video.
Generally, it consists of four steps:

\vspace{0.5cm}
\setstretch{1}
\begin{enumerate}
  \item Face detection
  \item Face normalization
  \item Facial feature extraction
  \item Classification
\end{enumerate}
\setstretch{\contentLineSpacing}

The first step of facial recognition is to identify all the faces in it. The biggest challenge of face detection
is that faces can be tilted and turned to any direction, and the program must be able to detect them regardless of
these situations. There are also some other factors that can prevent the faces from being correctly detected:
lighting, expressions, noises, face occlusions, etc. Secondly, the faces have to be normalized before performing
facial feature extraction. This step can include geometric normalization, lighting normalization, angle normalization.

Now we are ready to extract the unique features from a face, but questions arise: which features should we collect, and
how should we quantify a face? It can be the width between the eyes, the length of the nose, or even the width-height ratio
of the face. It turns out that the best approach is to let the computer itself figure this out by deep learning.
This is usually done by training a deep Convolutional Neural Network (CNN) which can generate a face encoding for each face
in the image. The last step is comparing this face encoding with other face encodings that already exist in the database.
If a match is found, then we've successfully identified an individual. The above process is actually similar to how we
identify people in real life:

\vspace{0.5cm}
\setstretch{1}
\begin{enumerate}
  \item Find all the faces within our vision.
  \item Identify the faces regardless of their angle, direction and lighting.
  \item Look for the unique features of the faces.
  \item Compare these unique features with all the people you know.
\end{enumerate}
\setstretch{\contentLineSpacing}


\subsection{Histogram of Oriented Gradients (HOG)}
Face Detection went popular in 2001 when Paul Viola and Michael Jones published the \emph{Viola–Jones object detection framework},
enabling objects on a camera to be detected in real time. Later on, more robust approaches are proposed. We will be using
Histogram of Oriented Gradients (HOG) in our system for this task.

Histogram of Oriented Gradients (HOG) is a feature description technique used in image processing with the aim of object
detection. The main idea is to split an image into several portions of the same size (since doing this for every pixel will give
us too much information more than we actually need), and then find out the major direction which brightness changes for each portion.
The result will be a representation of the fundamental structures of a face. Now we can look for the regions which are
similar to a known facial pattern generated from a huge amount of facial images. This way we can detect all the faces
in an image.


\subsection{Face Landmark Estimation}
Face Alignment is an important step in facial recognition. Before performing facial feature extraction, geometrically
normalizing faces can greatly improve the accuracy of facial recognition. We will attempt to locate 68 facial landmarks
by estimating their positions with an algorithm proposed by Vahid Kazemi and Josephine Sullivan in 2014 that is packaged
with dlib, and then use affine transformation to align the faces.


\subsection{Deep Metric Learning}
Previous researches have shown that deep learning does a better job than human in knowing which facial features should
be measured. Deep Metric Learning, a machine learning model, plays a vital role in modern facial recognition technology.
It works by training a deep Convolutional Neural Network (CNN) to generate 128-d measurements for each face. The training
process is typically carried out using a \emph{Triplet Network} which requires three images as input:

\vspace{0.5cm}
\setstretch{1}
\begin{enumerate}
  \item an image of person A
  \item another image of person A
  \item an image of person B
\end{enumerate}
\setstretch{\contentLineSpacing}

In each step of training, the algorithm will tweak the neural network, making the measurements of photo 1 and 2
slightly closer to each other while making the measurements of photo 2 and 3 slightly differenet from each other.
The training process only have to be done once and it has already been done by Davis King, the original author of dlib.
In addition, the trained neural network is publicly available so that we don't have to train the network ourselves.
We can directly use the pre-trained network to generate measurements for any face.

\subsection{SVM Classifier}
Face Detection went popular in 2001 when Paul Viola and Michael Jones published the \emph{Viola–Jones object detection framework},
enabling objects on a camera to be detected in real time. Later on, more robust approaches are proposed. We will be using
Histogram of Oriented Gradients (HOG) in our system for this task.
